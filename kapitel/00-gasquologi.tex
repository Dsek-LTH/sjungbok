\thispagestyle{empty}

\section{Gasquologi}
\subsection{Klädkod}
\subsubsection{Högtidsdräkt/högtidsklädsel}
För honom är frack lämplig. En fin golvlång klänning
gäller för henne---ett bra material och ett tjusigt snitt ska
det vara på den. Vill man bära handskar till klänningen
går det bra, men glöm inte att ta av dem när du äter.
Både han och hon kan istället bära hembygdsdräkt eller
militär högtidsdräkt.

\subsubsection{Smoking}
Då klädkoden är smoking, innebär det för honom en svart
eller mörkblå sådan. Förutom smokingjacka och byxa
ingår en särskild skjorta samt svart fluga. Hon har en lång
klänning, men den behöver inte vara lika elegant som vid
högtidsklädsel. Den får gärna vara festlig och glittrig.
Även kortare klänning---dock inte kortkort---går bra.

\subsubsection{Mörk kostym}
Han bär en mörkblå, mörkgrå eller svart kostym. Till det
har han en vit skjorta och sidenslips eller fluga i valfri färg
---gärna mönstrad! Hon har en klänning av siden, sammet
eller annat finare material. En elegant byxdress eller
halvlång kjol med jacka kan också bäras.

\subsubsection{Kavaj}
Tolkningen av kavaj kan variera. Om det är lite mer
högtidligt kan han bära mörk kostym. Är det mindre
formellt passar t.ex. en elegant blazer och udda byxa,
eller en ljusare kostym bra. Hon kan ha klänning, kjol eller
dress av varierande finhetsgrad beroende på
tillställningen. En hellång kjol eller klänning används inte.

\subsubsection{Vårdad klädsel}
Detta är egentligen ingen allmänt vedertagen klädkod,
men på LTH används den ibland. Det innebär i vilket fall
att man ska vara hel och ren, vilket många tolkar som
skjorta med snyggare byxor till. Undvik mjukisbyxor,
stövlar och saker som är jordiga.

\subsubsection{Ouveralle}
Som det låter. Bra att ha när man vill rulla i gräset och
dylikt. Bäres under mindre fina fester, även kallade
slasquer. Används mycket och gärna!

\newpage
\subsection{Etikett}

\subsubsection{Vid bordet}
Han har sin bordsdam till höger om sig och hon har sin
bordskavaljer till vänster. Herren drar ut stolen för sin
bordsdam då de sätter sig till bords. Vid bal förväntas det
ibland att man har present med sig till sin bordsdam eller
bordsherre.

\subsubsection{Skålen}
Han börjar till höger, sedan vänster och slutligen rakt
fram. Hon börjar till vänster, sedan höger och slutligen
rakt fram. Efter att man tagit en klunk tar man skålen
baklänges igen.

\subsubsection{Mat och dryck}
Vad som ingår eller hur många rätter som serveras
varierar på olika sittningar. Om vegetarisk mat önskas
samt om man har några allergier ska detta för det mesta
framföras i samband med att man köper biljett till
sittningen, och ofta kan man då även välja om man vill ha
alkohol till maten eller ej.
Ibland är baren öppen under själva sittningen, men det
händer även att vin, öl och snaps ingår i priset och att
baren hålls stängd. På vissa tillställningar säljs det snaps-
respektive punschbiljetter innan eller under sittningen för
den som så önskar. Håll utkik efter dessa!

\subsubsection{Sång och tal}
Sånger under en sittning dras igång av tillställningens
sångförmän och inte av enskilda gäster. Tycker man att
middagen börjar bli lite väl tråkig kan man dock begära
tempo. Då ser sångförmännen till att det sjungs en sång
eller hittas på något annat kul. Efter att ha varit på några
sittningar förstår man snart hur det går till. Om man ska
hålla tal eller hitta på något spex/gyckle ska man
underrätta sångförmännen i förväg så att de kan göra tid
för det under kvällen. Under tal, gycklen och skålar ska
man vara tyst, inte äta eller dricka. Det ingår i allmän
artighet att lyssna intresserat även om talaren själv håller
på att somna av sitt tal.


\subsubsection{Akademisk kvart}
Akademisk kvart, dvs. att evenemang börjar en kvart efter
utsatt tid, gäller normalt på LTH. Undantaget är tentor
som alltid börjar prick. Då det är helg eller kvällstid (dvs.
efter kl 18:00) gäller dubbelkvart, alltså 30 minuter efter
utsatt tid. Om det står ``prick'' (.) efter klockslaget,
``prickprick'' (..) då det gäller helgdag eller kvällstid, eller
om klockslaget skrivs med minutangivelse (till exempel klockan 13:00), så börjar
evenemanget på klockslaget.

\newpage
\subsubsection{Teknologmössan}
Teknologmössan får endast bäras av den som genomgått
TLTH:s nollning och kan därefter bäras oinskränkt.
Gäststudenter från utlandet är dock undantagna från
kravet på genomförd nollning. Det finns två olika varianter
av mössor; en vit sommarmössa som bärs mellan den 1
maj och 3 oktober och en mörkblå vintermössa som bärs
mellan 4 oktober och den 30 april. Vid högtidliga tillfällen
då högtidsdräkt bärs används dock alltid den vita mössan
även på vintern.
Det händer att den driftige teknologen under sin studietid
ackumulerar ett antal medaljer och pins. Dessa kan med
fördel pryda den blå mössan. Den vita mössan ska dock
hållas fri från dekorationer.



\subsubsection{Spegaten}
För varje påbörjat år på LTH sätter du en spegat i ditt
programs färg i tofsen på sin mössa. Dessutom finns det
ett antal specialspegater, bland annat en sorgsvart
spegat för uppehållsår, en skogsgrön om man tagit
uppehåll för att göra lumpen och en i vitt, blått och silver
som indikerar att man varit heltidare på Teknologkåren.

\subsubsection{Tofsen}
Tofsen innehåller hundratals trådar, och för att hålla
tofsen snygg så knyter man en knut längst ner på varje
tråd. Den som knyter en knut åt dig måste belönas med
en puss för slitgörat! En knut över spegaterna betyder att
mössinnehavaren är i ett förhållande. Det händer även att
tofsen bestyckas med mer eller mindre officiella
dekorationer för att påvisa sektions- eller kåraktivitet. Efter
avlagd examen fästs ett hänglås över spegaten för att
förhindra påtagandet av ytterligare spegater, nycklarna till
låset slängs i sjön Sjøn. Trula som tillhör D-chip bär en
elegant råsa råsett längst ner på sin tofs, under
spegaterna.
\newpage
\subsection{Rosa på bal}
\textit{Mel: Rosa på bal}\\
\index[alfa]{Rosa på bal}
\index[anfa]{Tänk att jag dansar med Andersson...}
\begin{parse lines}[\noindent]{#1\\}

Tänk att jag dansar med Andersson,
lilla jag, lilla jag,
med Fritiof Andersson!
Tänk att bli uppbjuden av en sån
populär person!

Tänk, vilket underbart liv, det Ni för!
Säg mig, hur känns det att vara charmör,
sjöman och cowboy, musiker, artist...
Det kan väl aldrig bli trist?

Nej, aldrig trist, fröken Rosa,
har man som Er kavaljer.
Vart jag än ställer min kosa,
aldrig förglömmer jag Er.

Ni är en sångmö från Helikons berg.
Åh, fröken Rosa, er linje, er färg,
skuldran, profilen med lockarnas krans,
ögonens varma glans!

Tänk, inspirera herr Andersson,
lilla jag, inspirera Fritiof Andersson!
Får jag kanhända min egen sång,
lilla jag, nån gång?

``Rosa på bal'', vackert namn, eller hur?
Början i moll och finalen i dur.
När blir den färdig, herr Andersson, säg,
visan Ni diktar till mig?

Visan om Er, fröken Rosa,
får Ni ikväll till Ert bord.
Medan vi talar på prosa
diktar jag rimmande ord.

Tyst! Ingen såg att jag kysste Er kind.
Känn hur det doftar från parken av lind!
Blommande lindar kring månbelyst stig...
Rosa, jag älskar dig!

\end{parse lines}
\vspace*{\fill}

\noindent {\textit{D-sektionens sektionshymn. Rosa sjunges alltid som färgen
    råsa. Rimmen modifieras därefter.}}

\newpage
\null

\newpage
